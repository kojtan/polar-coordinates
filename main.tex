
\documentclass[letterpaper,12pt]{article}
\usepackage[width=17cm]{geometry}
\usepackage[spanish]{babel}
\selectlanguage{spanish}
\usepackage[utf8]{inputenc}
\usepackage{mathrsfs}
\usepackage{enumerate}
\usepackage{mathtools}
\usepackage{pgf,tikz}
\usetikzlibrary{arrows}

\usepackage{multicol}
\setlength{\columnsep}{1cm}

\usepackage{mathptmx}
\usepackage[T1]{fontenc}


%\usepackage{mathpazo}
 \usepackage{amsmath,amsthm,amssymb,amsfonts, enumitem, fancyhdr, color, comment, graphicx, environ}
\pagestyle{fancy}
\setlength{\headheight}{65pt}

%\theoremstyle{definition}
\newtheorem{prob}{Problema}
\newtheorem{defi}{Definición}

%%%%%%%%%%%%%%%%%%%%%%%%%%%%%%%%%%%%%%%%%%%%%
%Fill in the appropriate information below
\lhead{\textbf{Matemática III}\\Hoja de problemas 4\\Semana 2 - Ciclo Impar 2020}  %replace with your name
\rhead{UES - F.CC.NN. y Mat.\\Escuela de Matemática\\por JuanJo Ramírez} %replace XYZ with the homework course number, semester (e.g. ``Spring 2019"), and assignment number.
%%%%%%%%%%%%%%%%%%%%%%%%%%%%%%%%%%%%%%%%%%%%%


%%%%%%%%%%%%%%%%%%%%%%%%%%%%%%%%%%%%%%
%Do not alter this block.
\begin{document}

\begin{center}
\Large \textbf{$\S$ Coordenadas polares}
\end{center}
\textbf{Indicación:} Resolver los siguientes problemas dejando claros los argumentos realizados para su resolución.

\begin{prob}
Los siguientes puntos dados están en coordenadas polares. Convertir a coordenadas cartesianas cada punto y graficar en el plano $XY$ ambas representaciones.
\begin{multicols}{3}
\begin{enumerate}[label=\alph*)]
    \item $(3,0)$
    \item $\displaystyle \left(4,\frac{\pi}{4}\right)$
    \item $(-5,0)$
    \item $\displaystyle \left(-4,\frac{\pi}{4}\right)$
    \item $\displaystyle \left(6,\frac{7\pi}{6}\right)$
    \item $\displaystyle \left(-3,-\frac{\pi}{6}\right)$
    \item $\displaystyle \left(-2,\frac{3\pi}{4}\right)$
    \item $(-5,\pi)$
    \item $\displaystyle \left(-3,\frac{5\pi}{4}\right)$
    \item $\displaystyle \left(\frac{3}{2},\frac{\pi}{6}\right)$
    \item $\displaystyle \left(-\frac{1}{3},\frac{5\pi}{3}\right)$
    \item $\displaystyle \left(-1,\frac{2\pi}{3}\right)$
    \item $\displaystyle \left(-2,-\frac{\pi}{3}\right)$
    \item $\displaystyle \left(-3,\frac{3\pi}{2}\right)$
    \item $\displaystyle \left(-4, \frac{\pi}{2}\right)$
    \item $\displaystyle \left(2,\frac{5\pi}{3}\right)$
\end{enumerate}
\end{multicols}
\end{prob}

\begin{prob}
Los siguientes puntos dados están en coordenadas polares. Convertir a coordenadas polares con $r>0$ y $0\leq \theta \leq 2\pi$ cada punto y graficar en el plano $XY$ ambas representaciones.
\begin{multicols}{3}
\begin{enumerate}[label=\alph*)]
    \item $(7,0)$
    \item $(2,2)$
    \item $(1,-1)$
    \item $(0,1)$
    \item $(0,-1)$
    \item $\left(2,2\sqrt{3}\right)$
    \item $(-2,2\sqrt{3})$
    \item $(-2\sqrt{3},2)$
    \item $(2,1)$
\end{enumerate}
\end{multicols}
\end{prob}

\begin{prob}
Demostrar que 
\[(-r,\theta)=(r,\theta+\pi)\;.\]
\emph{Sugerencia. Realizar un esbozo.}
\end{prob}

\begin{prob}
Describir el lugar geométrico de todos los puntos los cuales, en coordenadas polares, satisfacen la condición que $r=5$; hacer lo mismo para la condición $\displaystyle \theta=\frac{\pi}{3}$ y para la condición $\displaystyle \theta=-\frac{5\pi}{6}$. ¿Qué podemos decir del ángulo de intersección de la curva $r=k_1$, con $\theta=k_2$, donde $k_1,k_2\in\mathbb{R}$?
\end{prob}

\begin{prob}
Encontrar la distancia entre los puntos con coordenadas polares $\displaystyle \left(3,\frac{\pi}{4}\right)$, $\displaystyle \left(2,\frac{\pi}{3}\right)$.
\end{prob}

\begin{prob}
Encontrar la distancia entre los puntos con coordenadas polares $\displaystyle \left(1,\frac{\pi}{2}\right)$, $\displaystyle \left(4,\frac{5\pi}{6}\right)$.
\end{prob}

\begin{prob}
Encontrar una fórmula para la distancia entre los puntos $\displaystyle \left(r_1,\theta_1\right)$ y $\displaystyle \left(r_2,\theta_2\right)$.
\end{prob}

\begin{prob}
Encontrar la coordenada polar del punto medio del segmento de recta que conecta los puntos $P$ y $Q$ cuyas coordenadas polares son $\displaystyle P\left(3,\frac{\pi}{4}\right)$, $\displaystyle Q\left(6,\frac{\pi}{6}\right)$.
\end{prob}

\begin{prob}
Encontrar una fórmula para la coordenda $\displaystyle Q\left(\overline{r},\overline{\theta}\right)$ del punto medio del segmento de recta que conecta los puntos $P_1\left(r_1,\theta_1\right)$ y $P_2\left(r_2,\theta_2\right)$
\end{prob}
\end{document}